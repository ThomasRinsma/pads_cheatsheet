\documentclass[twocolumn,9pt]{extarticle}
\usepackage[english]{babel}
\usepackage[utf8]{inputenc}
\usepackage{amsmath}
\usepackage{listings}
\usepackage{palatino}
\usepackage[margin=0.5cm]{geometry}
\usepackage{enumitem}
\usepackage{amssymb}
\usepackage{braket}

%\setenumerate{noitemsep}
%\setlist{noitemsep}

\begin{document}

\begin{itemize}
	\setlength\itemsep{0.05cm}
	\item \textbf{Continuous pdf: } \\
	$f_X(x) : \mathcal{X} \rightarrow [0,\infty)$, $f_X(x) = \lim_{\epsilon \to 0} \frac{1}{\epsilon} \cdot \Pr[x \leq X \leq x + \epsilon]$

	\item \textbf{Continuous cdf: } \\
	$F_x(x) := \Pr[X \leq x]$ where $\Pr[a \leq X \leq b] = \int_a^b f_x(x) \, \mathrm{d} x$

	\item \textbf{Uniform distribution on $\mathcal{X} = [u,v]$: } \\
	$f_X(x) = 1/(v - u)$ for $x \in \mathcal{X}$

	\item \textbf{Normal distribution: } \\
	$f_X(x) = (2\pi)^{-1/2} \exp(-x^2/2)$

	\item \textbf{Dirac delta property: } \\
	$\int_{-\infty}^\infty \mathrm{d} x \, b(x)\delta(x-a) = b(a)$

	\item \textbf{Expectation value of $g(X)$ where $X \in \mathcal{X}$: } \\
	$\mathbb{E}[g(X)] = \sum_{x \in \mathcal{X}} \Pr[X = x] g(x)$
	, k'th moment of $X$= $\mathbb{E}[X^k]$

	\item \textbf{Statistical distance of $X, Y \in \mathcal{X}$: } \\
	$\Delta(X,Y) = \frac{1}{2} \sum_{x \in \mathcal{X}} |\mathbb{P}(x) - \mathbb{Q}(x)|$

	\item \textbf{Covariance matrix $K$: } \\
	$K_{i,j} = \mathbb{E}[X_i X_j] - \mathbb{E}[X_i] \cdot \mathbb{E}[X_j]$\\
	Zero covariance ($K_{1,2} = K_{2,1} = 0$) does not imply $X_1$ and $X_2$ are independent.

	\item \textbf{Marginal distribution for $X$ when $(X,Y) \sim \mathbb{P}$: } \\
	$\Pr[X = x] = \sum_y \mathbb{P}(x,y)$

	\item \textbf{Conditional probability for $(X,Y) \sim \mathbb{P}$: } \\
	$\Pr[X = x|Y = y] = \frac{\Pr[X = x, Y = y]}{\Pr[Y = y]} = \frac{\mathbb{P}(x,y)}{\mathbb{P}_2(y)}$

	\item \textbf{(Shannon) Entropy rules:}
		\begin{enumerate}
			\item \textbf{Additivity}: inf of a set of indep. RVs must be the sum of indiv. inf. contents 
			\item \textbf{Sub-additivity}: Total inf. content of two jointly distrib. RVs cannot exceed sum of seperate infs.
			\item \textbf{Expansibility}: Adding extra outcome of prob. 0 does not affect inf.
			\item \textbf{Normalization}: The distrib $(1/2,1/2)$ has inf. of 1 bit.
			\item The distrib $(p, 1-p)$ for $p \to 0$ has zero inf.
		\end{enumerate}

	\item \textbf{Shannon entropy:} \\
	Lower bound on the avg length of the shortest desc of $X$.\\
	$H(X) = \sum_{x \in \mathcal{X}} p_x \log_2 \frac{1}{p_x}$\\
	$H(X,Y) = \sum_{x \in \mathcal{X}} \sum_{y \in \mathcal{Y}} p_{xy} \log_2 \frac{1}{p_{xy}}$

	\item \textbf{Renyi entropy:} \\
	$H_\alpha(X) = \frac{-1}{\alpha - 1} \log \sum_{x \in \mathcal{X}} p^\alpha_x$

	\item \textbf{Binary entropy function: 2 outcomes with prob. $p$ and $1-p$: } \\
	$h(p) = p \log \frac{1}{p} + (1 - p) \log \frac{1}{1-p}$

	\item \textbf{Differential entropy for continuous RV $X \sim \rho$:} \\
	$h_{\text{diff}}(X) = - \int \mathrm{d}x\, \rho(x) \log \rho(x) = \mathbb{E}_x \log \frac{1}{\rho(x)}$

	\item \textbf{Relative entropy (Kullback-Leibler distance): } \\
	$D(\mathbb{P}||\mathbb{Q}) = \sum_{x \in \mathcal{X}} \mathbb{P}(x) \log \frac{\mathbb{P}(x)}{\mathbb{Q}(x)}$

	\item \textbf{Entropy of jointly distrib. RVs: $H(X,Y)$ or $H(XY)$: } \\
	$H(X,Y) = \sum_{x \in \mathcal{X}} \sum_{y \in \mathcal{Y}} p_{xy} \log \frac{1}{p_{xy}}$

	\item \textbf{Conditional entropy: } \\
	$H(X|Y) = \mathbb{E}_y[H(X|Y = y)] = - \sum_{x \in \mathcal{X}} p_x\, \sum_{y \in \mathcal{Y}} p_{x|y} \log p_{x|y}$ \\
	$H(X|Y) = H(X,Y) - H(Y)$
	
	\item \textbf{Mutual information: } \\
	$\mathbf{I}(X;Y) = H(X) - H(X|Y) = H(Y) - H(Y|X)$\\
	$\mathbf{I}(X;Y) = H(X,Y) - H(X|Y) - H(Y|X)$\\
	$\mathbf{I}(X;Y) = H(X) + H(Y) - H(X,Y)$\\
	$\mathbf{I}(X;Y|Z) = \mathbb{E}_z \mathbf{I}(X|Z=z ; Y|Z = z)$

	\item \textbf{Min entropy: } \\
	$H_{\text{min}}(X) = -\log \max\limits_{x \in \mathcal{X}} p_x = -\log p_{\text{max}}$\\
	$H_{\text{min}}(X|Y) = -\log \mathbb{E}_y \max\limits_{x \in \mathcal{X}} p_{x|y}$

	\item \textbf{Linear binary codes: } \\
	Maps $k$-bit msg $x$ to $n$-bit ($n > k$) codeword $c_x \in \mathcal{C}$. Perceived string $z = c_z \oplus e$. Minimum distance of code: $d = \min_{c,c'\in\mathcal{C}} \text{HammingWeight}(c \oplus c')$. Receiver determines which $c_{\hat{x}}$ is closest to $z$ and decodes it into $\hat{x}$. Error correcting capability $t = \lfloor\frac{d-1}{2}\rfloor$.

	\item \textbf{Generator ($G$ is $k \times n$) and parity check ($H$ is $(n-k) \times n$) matrix: } \\
	$c_x = xG$. $G = (\mathbf{1}_k|A)$. $H = (-A^T|\mathbf{1}_{n-k})$. $GH^T= 0$, $cH^t=0$.\\
	All $k$ rows of $G$ are linearly independent.

	\item \textbf{Syndrome decoding ($s(z) \in \{0,1\}^{n-k}$):} \\
	$s(z) = zH^T = (c_x + e)H^T = eH^T$\\
	Syndrome depends only on the error pattern, not on the message.

	\item \textbf{Hamming bound: Binary code of length $n$ that can correct $t$ errors: } \\
	$2^k \leq 2^n / \sum_{j=0}^t \binom nj$. Approx $\log n$ bits of redundancy per bit error.

	\item \textbf{Channel capacity:} \\
	Inf. content error free: $k \leq \mathbf{I}(C;Z)$\\
	BSC capacity (per bit): $\frac{k}{n} \leq \mathbf{I}(C_j;Z_j) = H(Z_j) - H(Z_j|C_j)$ \\
	This is called the BSC code rate: $\textsc{BSC code rate} \leq 1 - h(\epsilon)$ \\
	Following the rule of thumb: $h(\epsilon) = -\epsilon \log \epsilon + \mathcal{O}(\epsilon)$

	%\item \textbf{Uniformly random bits from continuous source:} \\
	%TODO

	\item \textbf{Randomness sources: } \\
	Ring oscilators (odd number of inverters causes jitter in period, gaussian), noisy resistors (no voltage applied, noise amplitude gaussian distribution), radioactive decay (poisson)

	\item \textbf{The von Neumann alg: } \\
	Given $(b_1, b_2)$, if $b_1 = b_2$ then no output, else output $b_1$.

	\item \textbf{Piling-up lemma: } \\
	Let $X_1, ..., X_n \in \{0,1\}$ be independent with biases $\Pr[X_i=1]-\Pr[X_i=0]=\alpha_i$. Construct $Y = X_1 \oplus X_2 \oplus ... \oplus X_n$. The bias of $Y$ is $\Pr[Y=1]-\Pr[Y=0] = (-1)^{n-1} \prod_{i=1}^n \alpha_i$. Thus by xoring many bits together the bias gets reduced.

	\item \textbf{Resilient function:} \\
	A function $\Psi : \{0,1\}^n \rightarrow \{0,1\}^m$ is $(n,m,t)$-resilient of, for any $t$ coords $i_1,...,i_t \in [n]$, any $a_1,...,a_t \in \{0,1\}$ and any $y \in \{0,1\}^m$ it holds that: $\Pr[\Psi(X)=y|x_{i_1}=a_1,...,x_{i_t}=a_t] = 2^{-m}$\\
	i.e.: Knowledge of $t$ values of the input does not give inf. that would help in guessing the output. ECC example ($[n,k,d]$ code): $\Psi : \{0,1\}^n \rightarrow \{0,1\}^k$. $\Psi = xG^T$. Then $\Psi$ is an $(n,k,d-1)$-resilient fun.

	\item \textbf{Strong extractor $\text{Ext} : \{0,1\}^n \times \{0,1\}^* \rightarrow \{0,1\}^l$:} \\
	Takes $n$-bit string $X$ and randomness $R$ and outputs an $l$-bit string $(l < n)$. $Z = \text{Ext}(X,R)$. Ext is a strong extractor for source min-entropy $m$, output length $l$ and nonuniformity $\epsilon$ if for all distrib of $X$ with $H_\infty(X) \geq m$ it holds that $\Delta(ZR;U_lR) \leq \epsilon$. $U_l$ is an RV uniform on $\{0,1\}^l$. Also true: $\mathbb{E}_r\Delta(Z|R=r;U_l)\leq\epsilon$

	%\item \textbf{Extractable randomness:}\\
	%TODO

	\item \textbf{Universal hash functions:} \\
	Let $\mathcal{R}$, $\mathcal{X}$ and $\mathcal{T}$ be finite sets. Let $\{\Phi_r\}_{r \in \mathcal{R}}$ be a family of hash functions from $\mathcal{X}$ to $\mathcal{T}$. The family is called universal iff, for $R$ drawn uniformly from $\mathcal{R}$, it holds that: $\Pr[\Phi_R(x) = \Phi_R(x')] \leq 1/|\mathcal{T}|$. It's called $\eta$-almost universal if it holds that: $\Pr[\Phi_R(x) = \Phi_R(x')] \leq \eta$ (for all $x, x' \in \mathcal{X}$ with $x \neq x'$).

	\item \textbf{Leftover hash lemma: } \\
	Let $X \in \mathcal{X}$ be a RV. Let $\delta \geq 0$ be a constant. Let $F : \mathcal{X} \times \mathcal{R} \rightarrow \{0,1\}^l$ be a $w^{-l}(1+\delta)$-almost universal family of hash functions, with seed $R \in \mathcal{R}$. Then: $\Delta(F(X,R)R;U_lR) \leq \frac{1}{2}\sqrt{\delta + 2^{l-H_2(X)}}$

	\item \textbf{Noisy broadcast channel and no return channel: } \\
	Secret capacity $C_s$ of the broadcast channel $P_{Y\,Z|X}$ can be bounded:\\
	$C_s(P_{Y\,Z|X}) \geq \max_{P_X}[\mathbf{I}(X;Y) - \mathbf{I}(X;Z)] = \max_{P_X}[H(X|Z) - H(X|Y)]$. Condition: if Eve's reception quality is better than Bob's ($\mathbf{I}(X;Y) < \mathbf{I}(X;Z)$) then the secrecy capacity is zero. Secrecy capacity of BSC with error rates $\epsilon$ and $\delta$ is: $h(\delta) - h(\epsilon)$ if $\delta > \epsilon$, and $0$ otherwise.

	\item \textbf{Noisy broadcast channel plus public return channel:} \\
	$\hat{C}_s(P_{Y\,Z|X}) \leq \min\{\max_{P_X}\mathbf{I}(X;Y), \max_{P_X}\mathbf{I}(X;Y|Z)\}$\\
	$\hat{C}_s(\epsilon,\delta) = h(\epsilon * \delta) - h(\epsilon)$.

	\item \textbf{Satellite scenario: } \\
	Alice and Bob agree on an ECC $\mathcal{C}$ with cw of length $N$. Alice chooses a random msg $R$, encodes it to $V^N$ and sends $V^N \oplus X^N$ to Bob over the noiseless pub channel (NPC). Bob computes $W^N = (V^N \oplus X^N) \oplus Y^N$. He accepts only if $W^N$ has much closer Hamming dist to some cw in $\mathcal{C}$ then the error correcting cap. of the code. He tells Alice over the NPC if he accepts or not. Noiseless for A en B but noisy for Eve: her noise is indep. so she has to guess at $X^N$ and hence at $R$.

	\item \textbf{PUF types and their properties: } \\
	General properties:
	\begin{itemize}
		\item The object can be subjected to a large number of diff challenges that yield an unpredictable response
		\item The object is very hard to clone physically
		\item Mathematical modeling of the challenge-response physics is very difficult
		\item Opaqueness: It is hard to characterize the physical structure of the object in a non-destructive way.
	\end{itemize}
	Types:
	\begin{itemize}
		\item \textbf{Coating PUF}: random layor of conductors and insulators: probes result in binary string. Used for secure key storage.
		\item \textbf{Optical PUF}: 3D optical structure produces speckle pattern. Challenge: props of laser beam: angle of incidence, focal dist.
		\item \textbf{Silicon PUF}: variations in IC from manufacturing. Challenge: pulsed time signal to certain part. Response: delay times of various wires and logic devices.
		\item \textbf{SRAM PUF}: Undefined state of RAM cells. Challenge: memory address, response: returned start-up values.
		\item \textbf{Randomly positioned glass fibers}: Challenge: ordinary beam of light lighting up part of the layer. Response: Certain fibers light up.
	\end{itemize}
	Uncontrolled PUF: reader interacts directly with PUF structure, trusted reader. Controlled PUF (CPUF): interaction through a \emph{control layer}, PUF and \emph{cl} are bound together, seperation will damage the PUF. Result: attacker has no direct access to PUF. Example: secure key storage where control layer performs zk-protocol to prove knowledge of the key (called \emph{Physically Obscured Key (POK)}).

	\item \textbf{PUF math:} \\
	Information revealed by a noisy meassurement outcome $U'$ where $m \in \mathcal{M} (m(K) = U)$: $\mathbf{I}(U';K) = \mathbf{I}(U';U)$. Noiseless case: $\mathbf{I}_m$\\

	Meassurable entropy of PUF (space $\mathcal{K}$ and $\mathcal{M}$):\\
	$\mathbf{I}^\text{meas}_{\mathbb{P}\mathcal{M}} = \max_{m\in\mathcal{M}} H(m(k))$\\

	Security param of bare PUF: min num of C-R meassurements required to reveal all measurable info of the PUF: $S_{\mathbb{P}\mathcal{M}_0}$

	\item \textbf{Fuzzy extractor: } \\
	Gen and Rep algorithms. $(S_x,W_x) = Gen(X)$. $S'_x = Rep(X', W_x)$
	Must satisfy the following properties:
	\begin{itemize}
		\item \textbf{Correctness}: The prob that $S'_x = S_x$ must be close to 1.
		\item \textbf{Security}: The RV $S_x$ must be close to uniform, given knowledge of $W_x$.
	\end{itemize}

	\item \textbf{Secure Sketch} (for discrete src space $\mathcal{X}$):\\
	$SS: x \mapsto w_x$, $Rec: (x', w_x) \mapsto \hat{x}$ with:
	\begin{itemize}
		\item \textbf{Correctness}: The prob that $\hat{X} = X$ must be close to 1.
		\item \textbf{Security}: $X$ given $W_X$ must have high entropy.
	\end{itemize}

	\item \textbf{When to use FE and SS: } \\
	FE: reliably extract a cryptographic key from noisy data. SS: reliably extract a string with sufficient (min-)entropy. Easier to construct SS than FE, in general: SS extracts more (min-)entropy than a FE from the same source.

	\item \textbf{Code offset method (COM):} \\
	Enroll (Gen): $s \in \{0,1\}^k$, $c_s = Enc(s)$. $w = c_S \oplus x$. Output $s$ as secret and $w$ as helper data.
	Reconstruct (Rep): $\hat{s} = Dec(x' \oplus w)$.

	% TODO: syndrome-only COM?

	\item \textbf{Zero leakage FE (for continuous RV) based on partitions } \\
	$\Pr[S=s|W=w] = 1/n$. Enroll: determine which partition the meassured val $x$ is located in: $\mathcal{A}_{ij}$. Set $s_x = i$, $w_x = j$. Reconstruct: $X'$ is meassured, determine for which $s'$ the interval $\mathcal{A}_{s',w_x}$ is closest to $x'$. This $s'$ is the reconstructed key. $H_\infty(S|W=w) = H_\infty(S|W) = \log n$.

	%\item \textbf{Helper data schemes for specific PUF types: } \\
	%TODO

	\item \textbf{Distance bounding principles (and fraud types): } \\
	\textbf{Mafia fraud}: challenge is relayed to different location where a legit. device is tricked into giving a response, response is relayed back to verifier.
	\textbf{Terrorist fraud}: legit. device cooperates with attacker, does not have to follow protocol, can share everything except long-term auth secrets.\\
	(Light travels about 300m every $\mu s$). If a device repeatedly correctly respons to an \emph{unpredictable} challenge within time $\Delta t$, then the location where the response is computed cannot be further away than $x = c \Delta t/2$. Max time for resp. to arrive: $t_{\text{max}} = 2 \frac{x_{\text{max}}}{c} + t_{\text{slack}}$. 

	\item \textbf{Brands-chaum protocol: } \\
	commit, rapid bit phase, sign phase. No link between phases, mafia not possible (timing meassured), terrorist is possible.
	\item \textbf{Swiss knife protocol } \\
	Rapid phase uses $R0_i=Z0_j$ and $R1_i=Z1_j$ to determine response $r_i$. $Z1 = Z0 \oplus x$ so attacker cannot perform rapid phase without knowledge of the key. Therefore terrorist and mafia are not possible. \textbf{Analog impl.} Similar to swiss knife but with LP and HP filters.

	%TODO: more about analog SK??

	\item \textbf{Linear algebra:} \\
	Complex conjugate of $a+bi$ is $a-bi$. conjugate of $\rho e^{i \phi}$ is $\rho e^{-i \phi}$. Hermitian conjugate of a complex number is its complex conjugate: $a^\dagger = a^*$.
	On a matrix: Hermitian conjugate is transpose followed by complex conjugate.
	The Hermitian conjugate of $\ket{\psi}$ is $\bra{\psi}$.

	\item \textbf{Quantum stuff:} \\
	\textbf{Meassurement destroys state information.}\\
	Time evolution of a quantum system can be represented as a unitary operator acting on a starting space. A unitary op $U$ is defined as: $UU^\dagger = \mathbf{1}$ and $U^\dagger U = \mathbf{1}$. Norm of a vec is preserved: For $\ket{\psi'} = U\ket{\psi}$, then the norm is $\braket{\psi'|\psi'} = \braket{\psi|U^\dagger U|\psi} = \braket{\psi|\psi}$.

	Tensor product:
	$\begin{pmatrix}\alpha \\ \beta\end{pmatrix} \otimes \begin{pmatrix}\gamma \\ \delta\end{pmatrix} = \begin{pmatrix}\alpha\gamma \\ \alpha\beta \\ \beta\gamma \\ \beta\delta\end{pmatrix}$.

	For qubits, tensor is omitted: $\ket{0} \otimes \ket{1} = \ket{01}$

	\item \textbf{No cloning theorem: } \\
	Let $\mathcal{H}_1$ and $\mathcal{H}_2$ be two Hilbert spaces. let $\ket{\psi} \in \mathcal{H}_1$ and $\ket{e} \in \mathcal{H}_2$, where $e$ is known and $\psi$ is unknown. Then there does not exist a unitary operator $U_e$ acting on $\mathcal{H}_1 \otimes \mathcal{H}_2$ satisfying $U_e\ket{\psi} \otimes \ket{e} = \ket{\psi} \otimes \ket{\psi}$ for all $\psi$


	\item \textbf{Quantum readout of PUFs:} \\
	One round of the protocol:
	\begin{enumerate}
		\item The verifyer choses a random challenge $\psi$. He prepares a particle in state $\ket{\psi}$ and sends the particle to the prover.
		\item The prover lets the particle interact with the PUF. This results in a state $\ket{\omega} =  R\ket{\psi}$. He sends the particle back to the verifyer.
		\item The verifier does a meassurement $\ket{\omega}\bra{\omega}$ on the particle. If the outcome is 1 then the prover has passed this round.
	\end{enumerate}

	\item \textbf{Shor:} \\
	Integer factorization. Classical part: Reduce factoring problem to order-finding problem. Quantum part: Quantum alg to solve order finding problem using the quantum Fourier transform.\\
	There is also a variant of Shor for solving discrete logarithms.

\end{itemize}

\end{document}